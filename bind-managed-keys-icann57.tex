\documentclass{beamer}
\usepackage{hyperref}
\title{BIND and root key rollover}
\author{Mukund Sivaraman\\
~\\
\small muks@isc.org}
\institute{Internet Systems Consortium}
\date{}
\usenavigationsymbolstemplate{}

\begin{document}

\frame{\titlepage}

\frame
{
  \frametitle{Default trust anchors in BIND}

  \begin{itemize}

  \item In BIND, the \textbf{bind.keys} file contains initial/starting
    trust anchors for the resolver for the \textbf{root zone}.

  \item When \textbf{dnssec-validation} is set to \textbf{yes}, no
    default trust anchor is used automatically. When
    \textbf{dnssec-validation} is set to \textbf{auto}, the keys in
    \textbf{bind.keys} file are used (for root only).

  \item For simpler configuration, a non-changeable copy of the default
    trust anchors is also built into the \textbf{named} program
    binary. If a \textbf{bind.keys} file exists, that will have
    precedence over the built-in copy.

  \item Non-root trust anchors need to be explicitly configured.

  \end{itemize}
}

\frame
{
  \frametitle{Manual trust anchor maintenance}

  \begin{itemize}
  \item The \textbf{trusted-keys} config option is used to introduce
    manually maintained trust anchors to \textbf{named}. Such trust
    anchors are not automatically updated.

  \item Without RFC 5011 feature, when the root key changes the root
    trust anchors would have to be updated manually, otherwise DNSSEC
    validation would fail.

  \item \textbf{bind.keys} (root zone) configures itself for RFC 5011
    automatic trust anchor maintenance (\textbf{managed-keys}).

  \end{itemize}
}

\frame
{
  \frametitle{Automatic trust anchor maintenance}

  \begin{itemize}

  \item RFC 5011 feature in BIND is known as \textbf{managed-keys} after
    the \textbf{named} config option. It was introduced in BIND 9.7.0.

  \item \textbf{bind.keys} provides initial/starting trust anchor
    configuration as \textbf{managed-keys} that have not been rolled. It
    is an input file that is \textbf{not modified} by \textbf{named}.

  \item \textbf{named} creates a corresponding
    \textbf{managed-keys.bind} or \textbf{viewname.mkeys} database file
    which contains keys in various states, including current trust
    anchors.

  \item After a root key rollover, the keys in \textbf{bind.keys} may
    become stale and invalid whereas the managed keys database is used
    for trust anchors.

  \end{itemize}
}

\frame
{
  \frametitle{Quirks}

  \begin{itemize}
    \item \textbf{named} uses format of master files to store the
      managed key database.
    \item Private RRTYPE code of 65533 is used to hold the key material
      and metadata.
    \item For new views, initial trust anchors will be taken from
      \textbf{bind.keys}, so a current copy should be provided by the
      admin.
  \end{itemize}
}

\frame
{
  \frametitle{Recommendations}

  \begin{itemize}
  \item Visit Warren Kumari's website \textbf{http://keyroll.systems/}
    for resources on testing key rollover.

  \item If you are setting up a new BIND resolver, update the
    \textbf{bind.keys} file to the latest copy as it provides the
    initial/starting root trust anchors (if the BIND installation
    pre-dates any root key rollovers).

  \item We publish the \textbf{bind.keys} file at
    \textbf{https://www.isc.org/bind-keys} and it will be updated when
    additional (future) root keys are available for distribution. The
    file also ships as part of the BIND source code and new releases
    will automatically have the latest copy of the file.
  \end{itemize}

}

\end{document}
