\documentclass{beamer}
\usepackage{hyperref}
\title{BIND and root key rollover}
\author{Mukund Sivaraman}
\institute{Internet Systems Consortium}
\date{}
\usenavigationsymbolstemplate{}

\begin{document}

\frame{\titlepage}

\frame
{
  \frametitle{Default trust anchors in BIND}

  \begin{itemize}

  \item In BIND, the \textbf{bind.keys} file contains initial/starting
    trust anchors for the resolver for the \textbf{root zone} and
    \textbf{DLV}.

  \item When \textbf{dnssec-validation} is set to \textbf{yes}, no
    default trust anchor is used automatically. When
    \textbf{dnssec-validation} is set to \textbf{auto}, the keys in
    \textbf{bind.keys} file are used (for root and DLV only).

  \item For simpler configuration, a non-changeable copy of the default
    trust anchors is also built into the \textbf{named} program
    binary. If an external \textbf{bind.keys} file exists, that will
    have precedence over the built-in copy.

  \item Non-root and non-DLV trust anchors need to be explicitly
    configured.

  \end{itemize}
}

\frame
{
  \frametitle{Manual trust anchor maintenance}

  \begin{itemize}
  \item The \textbf{trusted-keys} config option is used to introduce
    manually maintained trust anchors to \textbf{named}. Such trust
    anchors are not automatically updated.

  \item Without RFC 5011 feature, when the root key changes the root
    trust anchors would have to be updated manually, otherwise DNSSEC
    validation would fail.

  \item Similarly, trust anchors would have to be updated manually for
    any other starting points of trust (``security roots'') too.

  \item \textbf{bind.keys} (root and DLV) configures itself for RFC 5011
    automatic trust anchor maintenance (\textbf{managed-keys}).

  \item Other starting points of trust may use manual
    (\textbf{trusted-keys}) or automatic (\textbf{managed-keys}) trust
    anchor maintenance.

  \end{itemize}
}

\frame
{
  \frametitle{RFC 5011 in BIND}

  \begin{itemize}
  \item RFC 5011 feature in BIND is known as \textbf{managed-keys} after
    the \textbf{named} config option.
  \item \textbf{managed-keys} can be used for automatic trust anchor
    maintenance. A zone owner can add a ``stand-by" key to the
    security-root zone in advance. \textbf{named} would fetch and store
    the stand-by key, and when the original key was revoked,
    \textbf{named} would be able to transition smoothly to the new
    key. It would also recognize that the old key had been revoked, and
    cease using that key to validate answers.
  \item Implemented by Evan Hunt in June 2009.
  \item First released in BIND in 9.7.0.
  \item Bugs found and fixed over time.
  \item Some quirks remain.
  \end{itemize}
}

\frame
{
  \frametitle{Automatic trust anchor maintenance}

  \begin{itemize}

  \item \textbf{bind.keys} provides initial/starting trust anchors as
    \textbf{managed-keys} that have not been rolled. It is an input file that is
    \textbf{not modified} by \textbf{named}.

  \item \textbf{named} creates a corresponding
    \textbf{managed-keys.bind} or \textbf{viewname.mkeys} database file
    which contains the automatically managed keys database. This
    database file is maintained by \textbf{named} and has keys in
    various states, including current trust anchors. \textbf{named} does
    not use key material from \textbf{bind.keys} or any other
    \textbf{managed-keys} statements after creating the database, but
    uses them as a list of security roots.

  \item After a root key rollover, the keys in \textbf{bind.keys} may
    become stale and invalid. However, the resolver continues to work
    properly because the trust anchors from the managed keys database
    are used. This continues even after a restart of the resolver
    process.

  \end{itemize}
}

\frame
{
  \frametitle{Quirks}

  \begin{itemize}
  \item \textbf{Storing managed key database as zone files}
    \begin{itemize}
      \item \textbf{named} stores the managed key database in master
        files.
      \item Master file was used because it is an existing journaled
        serialization format in BIND.
      \item A private-use RRTYPE code 65533 is used for a
        \textbf{KEYDATA} RDATA type to represent managed trust anchor
        fields (refresh time, add hold-down time, remove hold-down time
        and DNSKEY fields) in such master files.
      \item In hindsight, this was not a good idea and if we were to
        implement it from scratch today, we'd use a real database
        library. Why was it not a good idea? :)
    \end{itemize}
  \item \textbf{Keeping \textbf{bind.keys} up to date when adding new views}
    \begin{itemize}
    \item After a root key rollover, the contents of \textbf{bind.keys}
      may become stale, though the resolver continues to work.
    \item When a new view is added, the resolver will try to use the
      keys from \textbf{bind.keys} as the initial/starting root trust
      anchors. So, it's important to check that the keys in
      \textbf{bind.keys} are current when adding a new view.
    \end{itemize}
  \end{itemize}
}

\frame
{
  \frametitle{Recommendations}

  \begin{itemize}
  \item Visit Warren Kumari's website \textbf{http://keyroll.systems/}
    for resources on testing key rollover.

  \item If you are setting up a new BIND resolver, update the
    \textbf{bind.keys} file to the latest copy as it provides the
    initial/starting root trust anchors (if the BIND installation
    pre-dates any root key rollovers).

  \item We publish the \textbf{bind.keys} file at
    \textbf{https://www.isc.org/bind-keys} and it will be updated when
    additional (future) root keys are available for distribution
    (expected in January 2017?). The file also ships as part of the BIND
    source code and new releases will automatically have the latest copy
    of the file.
  \end{itemize}

}

\end{document}
